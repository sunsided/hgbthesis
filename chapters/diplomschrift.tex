\chapter{Die Abschlussarbeit}
\label{cha:Diplomschrift}

Jede Abschlussarbeit%
\footnote{Die meisten der folgenden Bemerkungen gelten gleichsam für Bachelor-, Master- und Diplomarbeiten.} 
ist anders und dennoch sind sich gute
Arbeiten in ihrer Struktur meist sehr ähnlich, \va\ bei
technisch-natur\-wissen\-schaft\-lichen Themen. 

\section{Elemente der Abschlussarbeit}

Als Ausgangspunkt bewährt hat sich der folgende Grundaufbau, der natürlich 
vari\-iert und beliebig verfeinert werden kann:
%
\begin{enumerate}
\item \textbf{Einführung und Motivation}: Was ist die Problem- oder Aufgabenstellung und
warum sollte sich jemand dafür interessieren?
\item \textbf{Präzisierung des Themas}: Hier wird der aktuelle Stand der Technik
oder Wissenschaft ("`State-Of-The-Art"') beschrieben, es werden bestehende
Defizite oder offene Fragen aufgezeigt und daraus die
Stoßrichtung der eigenen Arbeit entwickelt.
\item \textbf{Eigener Ansatz}: Das ist natürlich der Kern der Arbeit. Hier
wird gezeigt, wie die vorher beschriebene Aufgabenstellung gelöst und --
häufig in Form eines Programms%
\footnote{\emph{Prototyp} ist in diesem Zusammenhang ein gerne benutzter Begriff, der im Deutschen
allerdings oft unrichtig dekliniert wird. Richtig ist: der \emph{Prototyp}, des \emph{Prototyps}, dem/den \emph{Protototyp} -- falsch hingegen \zB: des \emph{Prototyp\underline{en}}!
} --
realisiert wird, ergänzt durch illustrative Beispiele.
\item \textbf{Zusammenfassung}: Was wurde erreicht und welche Ziele sind
noch offen geblieben, wo könnte weiter gearbeitet werden?
\end{enumerate}
%
Natürlich ist auch ein gewisser dramaturgischer Aufbau der Arbeit
wichtig, wobei zu bedenken ist, dass der Leser in der Regel nur
wenig Zeit hat und -- anders als etwa bei einem Roman -- seine
Geduld nicht auf die lange Folter gespannt werden darf. Erklären
Sie bereits in der Einführung (und nicht erst im letzten Kapitel),
wie Sie an die Sache herangehen, welche Lösungen Sie vorschlagen
und wie erfolgreich Sie damit waren.

Übrigens, auch Fehler und Sackgassen dürfen (und sollten)
beschrieben werden; ihre Kenntnis hilft oft doppelte Experimente und
weitere Fehler zu vermeiden und ist damit sicher nützlicher als
jede Schönfärberei.
Und natürlich ist es auch nicht verboten, seine eigene Meinung 
in sachlicher Form zu äußern.


\section{Arbeiten in Englisch}
\label{sec:englisch}

Diese Vorlage ist zunächst darauf abgestellt, dass die
Abschlussarbeit in deutscher Sprache erstellt wird. Vor allem bei
Arbeiten, die in Zusammenarbeit mit größeren Firmen oder
internationalen Instituten entstehen, ist es häufig erwünscht,
dass die Abschlussarbeit zu besseren Nutzbarkeit in englischer
Sprache verfasst wird, und viele Hochschulen%
\footnote{Die FH Oberösterreich macht hier keine Ausnahme. 
Der Begriff "`Fachhochschule"' wird dabei entweder gar nicht
übersetzt oder -- wie im deutschsprachigen Raum mittlerweile üblich -- 
mit \emph{University of Applied Sciences}.
%Die offizielle englische Übersetzung von "`Medientechnik und -design"'
%ist übrigens \emph{Media Technology and Design}.
} 
lassen dies in
der Regel auch zu.

Beachtet sollte allerdings werden, dass das Schreiben dadurch nicht
einfacher wird, auch wenn einem Worte und Sätze im Englischen
scheinbar leichter "`aus der Feder"' fließen. Gerade im Bereich
der Informatik erscheint durch die Dominanz englischer
Fachausdrücke das Schreiben im Deutschen mühsam und das Ausweichen
ins Englische daher besonders attraktiv. Das ist jedoch
trügerisch, da die eigene Fertigkeit in der Fremdsprache
(trotz der meist langjährigen Schulbildung) häufig überschätzt wird.
Prägnanz und Klarheit gehen leicht verloren und bisweilen ist das
Resultat ein peinliches Gefasel ohne Zusammenhang und soliden
Inhalt. Sofern die eigenen Englischkenntnisse nicht wirklich gut sind, ist
es ratsam, zumindest die wichtigsten Teile der Arbeit zunächst in
Deutsch zu verfassen und erst nachträglich zu übersetzen. Besondere Vorsicht ist bei der Übersetzung von scheinbar
vertrauten Fachausdrücken angebracht. Zusätzlich ist es immer zu
empfehlen, die fertige Arbeit von einem "`native speaker"'
korrigieren zu lassen.



Technisch ist, außer der Spracheinstellung und den
unterschiedlichen Anführungszeichen (s.\
Abschn.~\ref{sec:anfuehrungszeichen}), für eine englische Arbeit
nicht viel zu ändern, allerdings sollte Folgendes beachtet werden:
%
\begin{itemize}
\item  Die Titelseite (mit der Bezeichnung "`Diplomarbeit"' oder "`Masterarbeit"') 
ist für die einzureichenden Exemplare jedenfalls in \emph{deutsch} zu halten,
auch wenn der Titel englisch ist. 
\item Ebenso muss neben dem
englischen \emph{Abstract} auch eine deutsche \emph{Kurzfassung}
enthalten sein. %
\item Akademische Titel von Personen haben im Englischen offenbar
weniger Bedeutung als im Deutschen und werden daher meist
weggelassen.
\end{itemize}
